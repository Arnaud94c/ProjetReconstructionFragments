\documentclass{article}
\usepackage[utf8]{inputenc}
\usepackage{indentfirst}
\usepackage{alltt}
\usepackage{graphicx}
\usepackage{amsmath,amsfonts,amssymb,amsthm}
\usepackage[left=2.7cm,right=2.7cm,top=2.5cm,bottom=2.5cm]{geometry}

\begin{document}
\title{Algorithmique et bioinformatique : Rapport de projet}
\author{Collin Arnaud, Galina Alicia}
\maketitle
\newpage
\tableofcontents
\newpage
\section{Introduction}
Le but de ce projet est de concevoir un programme d'assemblage de segments données pour fournir une séquence. Pour chaque collection de segments une séquence cible nous est fournie pour nous assurer de l'efficacité de notre programme. Celui-ci sera développé en JAVA et abordera des notions théoriques vues en cours.

\section{Explication de la démarche}

\subsection{Structures de données}
\subsubsection{CollectionFragments}

\subsubsection{Graphe}
Il représente le graphe de segments avec leurs scores d'alignement. Il contient 2 listes chaînées : une d'objet Node et l'autre d'objet Link. Il implémente aussi un algorithme de tri par tas qui classe les Link par ordre de scores décroissant.
\subsubsection{Node}
Il représente un noeud du graphe, il contient différentes informations :
\begin{itemize}
\item id : l'id du fragment (les fragments sont numérotés dans l'ordre d'apparition du fichier),
\item in : un booléen indiquant si nous sommes déjà rentré dans ce noeud (si son côté gauche est libre ou non),
\item out : un booléen indiquant si nous sommes déjà sorti de ce noeud (si son côté droit est libre ou non),
\item compl : un booléen indiquant si ce segment a déjà été choisi en complémentaire inversé.
\end{itemize}

\subsubsection{Link}
Il représente un lien du graphe, il contient différentes informations :
\begin{itemize}
\item sourceID : l'id du fragment source,
\item destinationID : l'id du fragment destination,
\item value : le score d'alignement de ce lien,
\item chaineSourceCompl : un booléen indiquant si le segment source est choisi en complémentaire inversé,
\item chaineDestinationCompl : un booléen indiquant si le segment destination est choisi en complémentaire inversé.
\end{itemize}


\subsection{Algorithmes}
\subsubsection{Lecture du fichier fasta}

\subsubsection{Semi-global}
Pour semi-global nous appliquons le même algorithme que vu en cours. Nous devons l'effectuer 2 fois par paire de fragments. En effet, il y a 8 façons d'arranger chaque paire. Il y a 2 formes possible pour chaque fragment : normal ou complémentaire inversé. Ce qui nous donne 4 combinaisons. Pour chaque combinaison, on peut les arranger de 2 manières différentes : segment $s$ suivi de segment $t$ ou l'inverse, ce qui donne également 2 scores différents. Nous avons donc au total bien 4x2 = 8 arrangements. \\

Pour une combinaison, l'algorithme semi-global nous donnera ces 2 scores. Besoin à priori de faire 4x semi-global. Or il existe des combinaisons qui donnent les mêmes scores : 
\begin{enumerate}
\item $s$ normal et $t$ normal = $s$ complémentaire inversé et $t$ complémentaire inversé.
\item $s$ normal et $t$ complémentaire inversé = $s$ complémentaire inversé et $t$ normal.
\end{enumerate}
Au final, nous avons bien besoin d'effectuer 2x semi-global pour nos 8 arrangements.

\subsubsection{Arrangement des liens}
Nous avons vu qu'il existe 8 arrangements. Cependant, nous savons que certains ont le même score. Il est inutile de stocker 8 liens quand seulement 4 pourraient être utilisés. En effet, pour un certain arrangement, nous savons quel autre arrangement lui est égale en score. Ce travail se fera au niveau de l'algorithme Greedy qui regardera quel arrangement peut convenir s'il y en existe un. 
\subsubsection{Greedy}
Nous trions d'abord les liens du graphe par score décroissant. Pour chaque lien, suivant sa catégorie, nous allons vérifier si une façon d'arranger les 2 fragments est acceptable, si oui alors on ajoute ce lien à notre chemin hamiltonien.
Lors de cette vérification il faut être prudent car pour chaque catégorie d'arrangements, nous avons 2 scores. Si c'est l'autre facon d'arranger le lien qui doit être prise alors nous créons un nouveau lien avec les bonnes caractéristiques pour le mettre dans notre chemin. Il faut dans ce cas échanger la source et la destination pour avoir le score de même valeur dans cette dispostion.
 

\section{Répartition des tâches}
\section{Points forts, points faibles et erreurs connues}
\section{Interprétation des résultats obtenus}
\section{Conclusion}

\end{document}
